\documentclass[11pt, a4paper]{article}
\usepackage{a4, fancyhdr}
\usepackage[latin1]{inputenc}
\usepackage[T1]{fontenc}
\usepackage{amsmath, amsthm, amssymb, eucal}
\usepackage{moreverb} %%encadrer du verbatim
\usepackage{graphicx}
\usepackage{hyperref}
\usepackage{framed}

%%\usepackage[top=20mm,bottom=20mm,left=25mm]{geometry}

\def\ec #1{\mathcal{#1}}
\def\mb #1{\mathbb{#1}}
\def\tt #1{\texttt{#1}}
\def\rm #1{\textrm{#1}}
\def\pbox #1{\framebox{\parbox{\textwidth}{#1}}}

%% \addtolength{\topmargin}{-2.5cm}
%% \setlength{\textheight}{25.5cm}

%% \addtolength{\textwidth}{1cm}


\usepackage{Sweave}
\begin{document}

% \setlength{\leftmargin}{-5cm}
% \setlength{\rightmargin}{-5cm}

\begin{center}
  {\huge \bf Estimate of dN and dS with Bio++ programs}\\
  \vskip 2em

\end{center}


\hrule

\medskip


We present a tutorial to estimate of dN and dS by normalized
stochastic mapping, with some programs developed on Bio++ libraries.


\section*{Programs}

Bio++ libraries are fully described on the wiki site:

\href{http://biopp.github.io}{http://biopp.github.io}

The methods presented here depend on two separate sets of programs,
\href{http://github.com/BioPP/bppsuite/}{\textbf{bppsuite}}
and \href{http://github.com/BioPP/testnh/}{\textbf{testnh}}.

This site explains how to install all libraries and programs.


\medskip

An apptainer image is available there:\\
\url{https://pbil.univ-lyon1.fr/bpp-doc/images/bpp_debian.sif}.

\medskip

All programs share the same syntax and options. In the next examples,
minimum configuration files are provided, but all options of the
programs are described in the
\href{https://pbil.univ-lyon1.fr/bpp-doc/bppsuite/bppsuite.html}{bppsuite
  manual} and in the
\href{http://biopp.github.io/testnh/}{testnh
  manual}.



\section*{Estimate on simulated sequences}


\subsection*{Simulating sequence evolution}

The program to simulate the evolution of sequences is
\texttt{bppseqgen}, and we can call like this:

\begin{verbatim}
bppseqgen THETA_ROOT=0.7 THETA_EQ=0.3 param=seq.bpp
\end{verbatim}

In the configuration file \texttt{seq.bpp}, one unique model is used
on all the branches of the tree described in
file \texttt{tree.dnd}. 

The model used is the standard model of Yang \& Nielsen for codons
(see
\href{https://pbil.univ-lyon1.fr/bpp-doc/bpp-phyl/html/classbpp_1_1YN98.html#details}{the
  bio++ link} for a description of the parametrization), with F1X4
parametrization of the codon distribution. \verb|THETA_ROOT| is the GC
content at the root of the tree, and \verb|THETA_EQ| is the
equilibrium GC content. The proportions of C and G are equal, as well
as the proportions of A and T.

The output alignment is in file \texttt{simul\_seq.fa}.


\subsection*{Maximum likelihood estimate of model and branch lengths}

From a given alignment and a given topology, we estimate the most
likely model and branch lengths with program \texttt{bppml}:

\begin{verbatim}
bppml SEQ=simul_seq.fa param=ml.bpp
\end{verbatim}

\texttt{tree\_init.dnd} is the starting tree for the optimization
procedure. The only important feature in this tree is the topology,
since it is not optimized. The branches are numbered in the order they
appear in the file, for example here:
\texttt{(((0,(1,2)3)4,(5,6)7)8,9);} (for user declaration, use Nhx
format (see bppsuite manual)). The optimized tree and model are
respectively in files \texttt{tree\_ml.dnd} and
\texttt{model\_ml.params}.

\subsection*{Estimates of dN and dS}

Finally, to compute the normalized expected dN and dS of the tree, we
use \texttt{mapnh} with the optimal tree and model obtained
previously. Several outputs are available for the estimates of dN and
dS.

\subsubsection*{Estimates per branch}

We get one tree file per type of substitution, with branch lengths
replaced by the branch specific estimates.

\begin{verbatim}
mapnh SEQ=simul_seq.fa MODEL=model_ml TREE=tree_ml \
                       param=map_per_branch.bpp
\end{verbatim}

The dN and dS counts on all branches are respectively in tree files
\\ 
\verb|simul_seq.fa.counts_dN.dnd| and
\verb|simul_seq.fa.counts_dS.dnd|.

\smallskip

Here is an example of usage in R :

\begin{verbatim}
library(ape)

treedN=read.tree(file="simul_seq.fa.counts_dN.dnd")  
treedS=read.tree(file="simul_seq.fa.counts_dS.dnd")  

plot(treedS)

## length of the edges 
treedN$edge.length
treedS$edge.length

## primate edges are numbered 2, 3, 4, 5, 6
## dN and dS on these edges

sum(treedN$edge.length[2:6])
sum(treedS$edge.length[2:6])

## estimated dN/dS on the primate clade

sum(treedN$edge.length[2:6])/sum(treedS$edge.length[2:6])

\end{verbatim}


\subsubsection*{Counting per site}

It is also possible to get the estimates per site:

\begin{verbatim}
mapnh SEQ=simul_seq.fa MODEL=model_ml TREE=tree_ml \
                       param=map_per_site.bpp
\end{verbatim}


The output is in file \verb|simul_seq.fa_sites.count.sged|, with a column
for dS and a column for dN.

\subsubsection*{Counting per site per branch}

It is also possible to get the estimates per site per branch:

\begin{verbatim}
mapnh SEQ=simul_seq.fa MODEL=model_ml TREE=tree_ml \
                       param=map_per_site_per_branch.bpp
\end{verbatim}

The dN and dS counts on all branches and sites are respectively in
files \verb|simul_seq.fa_branch_sites_dN.count| and\\
\verb|simul_seq.fa_branch_sites_dS.count|, with columns numbered
with branch numbers.


\section*{Estimate with stationarity assumption}

We perform similar estimates, but with the assumption of
stationarity:

\begin{verbatim}
bppml SEQ=simul_seq.fa param=ml_stat.bpp

mapnh SEQ=simul_seq.fa MODEL=model_stat_ml TREE=tree_stat_ml \
                       PREF=_stat param=map_per_branch.bpp

\end{verbatim}

And as before, in R:

\begin{verbatim}

treedNstat=read.tree(file="simul_seq.fa_stat.counts_dN.dnd")  
treedSstat=read.tree(file="simul_seq.fa_stat.counts_dS.dnd")  

## length of the edges 
treedNstat$edge.length
treedSstat$edge.length

## primate edges are numbered 2, 3, 4, 5, 6
## dN and dS on these edges

sum(treedNstat$edge.length[2:6])
sum(treedSstat$edge.length[2:6])

## estimated dN/dS on the primate clade

sum(treedNstat$edge.length[2:6])/sum(treedSstat$edge.length[2:6])

\end{verbatim}


\section*{Simulation and estimate in non-homogeneous modeling}

Here is an example of simulation and estimate with non-homogeneous
modeling. One model is used for the branches in the primate clade
(numbers 0 to 4), and another one for the other branches (numbers 5 to
9).

To reduce the number of parameters for the estimate, we assume that
$\omega$ as well as equilibrium proportions of C/(C+G) and A/(A+T) are
equal in both models.

\begin{verbatim}
bppseqgen THETA_ROOT=0.7 THETA_EQ_PRIM=0.7 THETA_EQ_OTHER=0.3 \
                         param=seq_nonhom.bpp

bppml SEQ=simul_seq_nonhom.fa param=ml_nonhom.bpp

mapnh SEQ=simul_seq_nonhom.fa MODEL=model_nonhom_ml \
      TREE=tree_nonhom_ml param=map_per_branch.bpp
 
\end{verbatim}

And in R:

\begin{verbatim}

treedNnonhom=read.tree(file="simul_seq_nonhom.fa.countsdN.dnd")  
treedSnonhom=read.tree(file="simul_seq_nonhom.fa.countsdS.dnd")  

## length of the edges 
treedNnonhom$edge.length
treedSnonhom$edge.length

## primate edges are numbered 2, 3, 4, 5, 6
## dN and dS on these edges

sum(treedNnonhom$edge.length[2:6])
sum(treedSnonhom$edge.length[2:6])

## estimated dN/dS on the primate clade

sum(treedNnonhom$edge.length[2:6])/sum(treedSnonhom$edge.length[2:6])

\end{verbatim}


\end{document}



