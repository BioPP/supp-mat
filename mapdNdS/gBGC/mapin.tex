\documentclass[11pt, a4paper]{article}
\usepackage{a4, fancyhdr}
\usepackage[latin1]{inputenc}
\usepackage[T1]{fontenc}
\usepackage{amsmath, amsthm, amssymb, eucal}
\usepackage{moreverb} %%encadrer du verbatim
\usepackage{graphicx}
\usepackage{hyperref}
\usepackage{framed}

%%\usepackage[top=20mm,bottom=20mm,left=25mm]{geometry}

\def\ec #1{\mathcal{#1}}
\def\mb #1{\mathbb{#1}}
\def\tt #1{\texttt{#1}}
\def\rm #1{\textrm{#1}}
\def\pbox #1{\framebox{\parbox{\textwidth}{#1}}}

%% \addtolength{\topmargin}{-2.5cm}
%% \setlength{\textheight}{25.5cm}

%% \addtolength{\textwidth}{1cm}


\usepackage{Sweave}
\begin{document}

% \setlength{\leftmargin}{-5cm}
% \setlength{\rightmargin}{-5cm}

\begin{center}
  {\huge \bf Estimation of dN and dS considering different substitution categories}\\
  \vskip 2em

\end{center}

\hrule

\medskip


We present a tutorial to estimate dN and dS for different categories of
substitutions, either overall, or depending on the change in GC
content.

\texttt{S} (resp. \texttt{W}) stands for \texttt{G} or \texttt{C}
(resp. \texttt{A} or \texttt{T}). And, for example, \texttt{S} $
\rightarrow $ \texttt{W} is any substitution from \texttt{S} to
\texttt{W}, and \texttt{dS X S} $ \rightarrow $ \texttt{W} is any
synonymous such substitution.


\subsection*{Maximum likelihood estimate of model and branch lengths}

The first step is to infer the best-fit parameters for a specific model, root and tree
from the data by maximum likelihood. Here, we use the YN98 model.

\begin{verbatim}
bppml param=base.bpp
\end{verbatim}

\texttt{TENT.dnd} is the starting tree for the optimization procedure.
The only important feature in this tree is the topology, which is not
optimized. The branches are numbered in the order they appear in the
file, for example here: \texttt{(((0,(1,2)3)4,(5,6)7)8,9);} (for user
declaration, use Nhx format (see bppsuite manual)). The optimized tree
and model are respectively in files \texttt{tree\_ml.dnd} and
\texttt{model\_ml.params}\footnote{In the configuration file we set
  the tolerance to stop the optimization process rather high,
  \texttt{optimization.tolerance = 10}, to shorten the computation
  time. The user can check that with lower values the estimates of the
  parameters are not much different.}.

\subsection*{Estimates of dN and dS}

Then, to compute dN and dS on each branch, we use \texttt{mapnh} with
the optimal tree and model obtained previously.

We get one tree file per type of substitution, with branch lengths
replaced by the branch specific estimates.

\begin{verbatim}
mapnh param=map_dNdS.bpp
\end{verbatim}

The dN and dS counts on all branches are respectively in tree files
\\ 
\verb|TENT.counts_dN.dnd| and
\verb|TENT.counts_dS.dnd|.

\smallskip

An example of usage in R is in file \texttt{manip.R}. In this file, we
use the library \texttt{ape} to handle trees. Note that the nodes
numbers are different between \texttt{bio++} and \texttt{ape}.

\subsection*{Category specific estimates of dN and dS}

With command

\begin{verbatim}
mapnh param=map_dNdS_GC.bpp
\end{verbatim}

we compute dN and dS for each category \texttt{S} $
\rightarrow $ \texttt{W}, \texttt{S} $
\rightarrow $ \texttt{S}, \texttt{W} $
\rightarrow $ \texttt{W}, \texttt{W} $
\rightarrow $ \texttt{S}.

The dN and dS counts for the different categories on all branches
are found in the tree files, such as \\ \verb|TENT.counts_dN_X_W->S.dnd| and
\verb|TENT.counts_dS_X_W->S.dnd|.

\medskip

Foregoing example of usage in R is still in file \texttt{manip.R}.
% I don't understand what you mean by this last sentence?



\end{document}
